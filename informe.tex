\documentclass[12pt,a4paper]{article}
\usepackage[latin1]{inputenc}
\usepackage[spanish]{babel}
\usepackage{amsmath}
\usepackage{amsfonts}
\usepackage{amssymb}
\usepackage{graphicx}
\usepackage[left=2cm,right=2cm,top=2cm,bottom=2cm]{geometry}
\newtheorem{mydef}{Definici�n}
\author{Romina}
\title{Introduccion a Latex}
\begin{document}
\maketitle
%\makeindex

\section{Intoduccion}
Esta es la introducci�n del trabajo de \textit{Proyecto RLC}.

\section{Matrices}

\begin{equation*}
I_{3}= 
\begin{pmatrix}
1	&	0	&	0\\
0	&	1	&	0\\
0	&	0	&	1
\end{pmatrix}
\end{equation*}

\section{Ecuaciones Diferenciales}

\begin{mydef}[Operador Nabla]
El operador nabla de la funci�n $f$.
\begin{equation}
\bigtriangledown f = \sum_{i=1}^{n}\overrightarrow{e_{i}}\frac{\partial f}{\partial x_{i}}
\end{equation}

\end{mydef}
\section{Figuras}
\end{document}